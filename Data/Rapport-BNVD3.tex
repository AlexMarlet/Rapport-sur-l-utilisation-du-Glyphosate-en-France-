% Options for packages loaded elsewhere
\PassOptionsToPackage{unicode}{hyperref}
\PassOptionsToPackage{hyphens}{url}
%
\documentclass[
]{article}
\usepackage{amsmath,amssymb}
\usepackage{iftex}
\ifPDFTeX
  \usepackage[T1]{fontenc}
  \usepackage[utf8]{inputenc}
  \usepackage{textcomp} % provide euro and other symbols
\else % if luatex or xetex
  \usepackage{unicode-math} % this also loads fontspec
  \defaultfontfeatures{Scale=MatchLowercase}
  \defaultfontfeatures[\rmfamily]{Ligatures=TeX,Scale=1}
\fi
\usepackage{lmodern}
\ifPDFTeX\else
  % xetex/luatex font selection
\fi
% Use upquote if available, for straight quotes in verbatim environments
\IfFileExists{upquote.sty}{\usepackage{upquote}}{}
\IfFileExists{microtype.sty}{% use microtype if available
  \usepackage[]{microtype}
  \UseMicrotypeSet[protrusion]{basicmath} % disable protrusion for tt fonts
}{}
\makeatletter
\@ifundefined{KOMAClassName}{% if non-KOMA class
  \IfFileExists{parskip.sty}{%
    \usepackage{parskip}
  }{% else
    \setlength{\parindent}{0pt}
    \setlength{\parskip}{6pt plus 2pt minus 1pt}}
}{% if KOMA class
  \KOMAoptions{parskip=half}}
\makeatother
\usepackage{xcolor}
\usepackage[margin=1in]{geometry}
\usepackage{graphicx}
\makeatletter
\def\maxwidth{\ifdim\Gin@nat@width>\linewidth\linewidth\else\Gin@nat@width\fi}
\def\maxheight{\ifdim\Gin@nat@height>\textheight\textheight\else\Gin@nat@height\fi}
\makeatother
% Scale images if necessary, so that they will not overflow the page
% margins by default, and it is still possible to overwrite the defaults
% using explicit options in \includegraphics[width, height, ...]{}
\setkeys{Gin}{width=\maxwidth,height=\maxheight,keepaspectratio}
% Set default figure placement to htbp
\makeatletter
\def\fps@figure{htbp}
\makeatother
\setlength{\emergencystretch}{3em} % prevent overfull lines
\providecommand{\tightlist}{%
  \setlength{\itemsep}{0pt}\setlength{\parskip}{0pt}}
\setcounter{secnumdepth}{-\maxdimen} % remove section numbering
\usepackage{booktabs}
\usepackage{longtable}
\usepackage{array}
\usepackage{multirow}
\usepackage{wrapfig}
\usepackage{float}
\usepackage{colortbl}
\usepackage{pdflscape}
\usepackage{tabu}
\usepackage{threeparttable}
\usepackage{threeparttablex}
\usepackage[normalem]{ulem}
\usepackage{makecell}
\usepackage{xcolor}
\usepackage{caption}
\usepackage{anyfontsize}
\ifLuaTeX
  \usepackage{selnolig}  % disable illegal ligatures
\fi
\usepackage{bookmark}
\IfFileExists{xurl.sty}{\usepackage{xurl}}{} % add URL line breaks if available
\urlstyle{same}
\hypersetup{
  pdftitle={Les détérminents de la consommation de glyphosate en France},
  pdfauthor={Caleb Colin / Brice Perroud / Lucas Gerardy France / Alexandre Marlet},
  hidelinks,
  pdfcreator={LaTeX via pandoc}}

\title{Les détérminents de la consommation de glyphosate en France}
\author{Caleb Colin / Brice Perroud / Lucas Gerardy France / Alexandre
Marlet}
\date{2025-03-30}

\begin{document}
\maketitle

\section{Introduction}\label{introduction}

Les pesticides prennent une place importante dans les productions
agricoles en france pour faire face à une concurrence internationale.

Problématique: Existe-t-il une corrélation entre la répartition des
types de cultures par département (céréale, viticulture) et la
production brut standard ainsi que la surface agricole utilisé (pbs et
sau) et la consommation entre les différentes substances
phytosanitaires, l'effet d'une taxe serait-il significatif ?

Autrement dit, par l'intermédiaire de la consommation par département de
glyphosate et de la culture dominante, la production prut standard et la
surface unitaire utile par département, nous chercherons à expliqué si
une culture à un impact significatif sur la consomation de glyphosate et
chercheront si cette relation est d'avantage significative en fonction
de la pbs ou de la sau Aussi nous mesurerons l'impacte d'une taxe sur
l'utilisation de glyophosate.

On se servira pour notre étude des données du recencement agricole de
2020 qui sont les dernière données récolté, et nous les agregerons avec
les données de consommation de glyphosate de 2022.

\subsection{Sources des données}\label{sources-des-donnuxe9es}

substance
(\url{https://www.data.gouv.fr/fr/datasets/achats-de-pesticides-par-code-postal/})

vignes
(\url{https://agreste.agriculture.gouv.fr/agreste-web/disaron/Carte-RA-partvigne20/detail/})

pbs
(\url{https://agreste.agriculture.gouv.fr/agreste-web/disaron/Carte-RA-pbs20/detail/})

sau
(\url{https://agreste.agriculture.gouv.fr/agreste-web/disaron/Carte-RA-sau20/detail/})

cereales
(\url{https://agreste.agriculture.gouv.fr/agreste-web/disaron/Carte-RA-partcereoleopro20/detail/})

\subsection{Chargement et fusion des bases de
données}\label{chargement-et-fusion-des-bases-de-donnuxe9es}

Nous fusionnons les bases de données pour obtenir une seul et même base
contenant les informations utiles. Ces bases sont réunis par la variable
en commun ``code departement''. Les information manquante représenté par
le symbole N/A seront comptabilisé comme une valeur nulle c'est à dire
``0''.

\section{Description des bases de
données}\label{description-des-bases-de-donnuxe9es}

Nous travaillons sur 5 bases de données que nous fusionnons pour étudier
la consommation de substance phytosanitaires par département en fonction
des culture sur des obsrevations de 2022.

\section{Dictionnaire des variables}\label{dictionnaire-des-variables}

code\_departement - numéro indicatif pour chaque département

departement - nom du département

qte\_glyphosate - quantité de glyphosate achetée dans le département
(exprimé en Kg)

cereales - part des cereales et oléagineux dans la sau (en \%)

vignes - part de vignes dans la sau (en \%)

pbs - production brut standard, production potentielle totale des
exploitations par département, résultent des valeurs moyennes des
rendements et des prix observés sur la période 2015 à 2019, exprimé en
euros.

log\_pbs - le logarithme de la variable pbs

sau - superficie agricole utilisée (en ha.), comprenant les céréales,
les oléagineux, protéagineux et plantes à fibres, les autres plantes
industrielles destinées à la transformation, les cultures fourragères et
les surfaces toujours en herbe, les légumes secs et frais, les fraises
et les melons, les pommes de terre, les fleurs et plantes ornementales,
les vignes, les autres cultures permanentes (vergers, petits fruits,
pépinières ligneuses), les jachères, les jardins et vergers familiaux.

log\_sau - le logarithme de la variable sau

\section{Statistiques descriptives}\label{statistiques-descriptives}

\subsection{Statistiques univariées}\label{statistiques-univariuxe9es}

\begin{verbatim}
##     cereales        vignes               pbs               sau        
##  Min.   : 0.00   Length:92          Min.   :    103   Min.   :     1  
##  1st Qu.:13.55   Class :character   1st Qu.: 337547   1st Qu.:143522  
##  Median :37.90   Mode  :character   Median : 544628   Median :291324  
##  Mean   :37.54                      Mean   : 670574   Mean   :269655  
##  3rd Qu.:59.48                      3rd Qu.: 909984   3rd Qu.:385687  
##  Max.   :84.90                      Max.   :2685717   Max.   :557179  
##  qte_glyphosate        log_pbs          log_sau     
##  Min.   :     1.8   Min.   : 4.635   Min.   : 0.00  
##  1st Qu.: 15894.4   1st Qu.:12.729   1st Qu.:11.87  
##  Median : 53155.4   Median :13.208   Median :12.58  
##  Mean   : 59294.4   Mean   :12.896   Mean   :11.95  
##  3rd Qu.: 95079.0   3rd Qu.:13.721   3rd Qu.:12.86  
##  Max.   :199723.1   Max.   :14.803   Max.   :13.23
\end{verbatim}

On peut déduire de ce tableau que la quantité moyenne de Glyphosate
acheté par département est de \textbf{``a compléter quand on aura les 9
departements qui manques''} Kg

\subsection{Statistiques bivariées}\label{statistiques-bivariuxe9es}

\subsubsection{Correlations entre les
variables}\label{correlations-entre-les-variables}

\begin{center}\includegraphics[width=0.8\linewidth,]{Rapport-BNVD3_files/figure-latex/unnamed-chunk-6-1} \end{center}

Nous avons enlevé la diagonale puisque la correlation d'une variable
avec elle même est de 1.

\section{Modèle d'éstimation}\label{moduxe8le-duxe9stimation}

\[\log(\text{Quantite}) = \text{const} + \beta_1 \log(\text{sau}_i) + \beta_2 \log(\text{pbs}) + \beta_3 (\text{cereales}) + \beta_4 (\text{vignes}) + u\]

On cherche à expliquer la quantité de glyphosate en fonction de la
surface exploitée, de la production brute standard, de la part de
céréale dans l'exploitation, de la part de vignes dans l'exploitation.
Le but étant de determiner si la surface exploité a un impact
significatif sur la quantité de substance utilisé (ici le Glyphosate),
ainsi que determiner si la culture que ce soit céréale ou vignes a un
impact sur la consommation de glyphosate ( savoir, si une culture est en
moyenne plus consommatrice de glyphosate que les autres).

\subsection{MCO1}\label{mco1}

\begin{table}[!t]
\fontsize{12.0pt}{14.4pt}\selectfont
\begin{tabular*}{\linewidth}{@{\extracolsep{\fill}}lc}
\toprule
  & (1) \\ 
\midrule\addlinespace[2.5pt]
(Intercept) & 6.243*** \\ 
 & (1.750) \\ 
log\_sau & 0.330 \\ 
 & (0.227) \\ 
log\_pbs & -0.155 \\ 
 & (0.311) \\ 
cereales & 0.051*** \\ 
 & (0.007) \\ 
vignes & 0.053** \\ 
{} & {(0.017)} \\ 
Num.Obs. & 80 \\ 
R2 & 0.532 \\ 
R2 Adj. & 0.507 \\ 
AIC & 1939.8 \\ 
BIC & 1954.1 \\ 
Log.Lik. & -137.302 \\ 
F & 21.315 \\ 
RMSE & 1.35 \\ 
\bottomrule
\end{tabular*}
\begin{minipage}{\linewidth}
+ p < 0.1, * p < 0.05, ** p < 0.01, *** p < 0.001\\
\end{minipage}
\end{table}

On remarque qu'uniquement les variables ``cereales'' et ``vignes'' sont
signivicatives, la première au seuil de 95\% et la deuxieme au seuil de
99\%.

\subsubsection{Qualité d'ajustement}\label{qualituxe9-dajustement}

\paragraph{Test de Fisher}\label{test-de-fisher}

\begin{verbatim}
## Test de Fisher pour la significativité globale du modèle :
##  Statistique F = 21.31 
##  Degrés de liberté (modèle) = 4 
##  Degrés de liberté (résidus) = 75 
##  p-value = < 0.001
\end{verbatim}

\begin{longtable}[t]{lcccc}
\caption{\label{tab:unnamed-chunk-8}Test de Fisher - Significativité globale du modèle}\\
\toprule
 & Statistique.F & ddl1 & ddl2 & p.value\\
\midrule
value & 21.31 & 4 & 75 & < 0.001\\
\bottomrule
\end{longtable}

Le modèle est globalement significatif.

\paragraph{Analyse du R2}\label{analyse-du-r2}

Pour ce qui est du R2 il est 0.53 c'est à dire que 53\% de la variance
est expliquée par le modèle. C'est très peu, cela signifie que 47\% de
la variance dépend d'autre facteurs non pris en compte dans le modèle.

\subsubsection{Test d'hétéroscedasticité (Hypothèse
H2)}\label{test-dhuxe9tuxe9roscedasticituxe9-hypothuxe8se-h2}

\paragraph{Visualisation de
l'hétéroscedasticité}\label{visualisation-de-lhuxe9tuxe9roscedasticituxe9}

\begin{center}\includegraphics[width=0.8\linewidth,]{Rapport-BNVD3_files/figure-latex/unnamed-chunk-9-1} \end{center}

\paragraph{Test par la methode de Breush
Pagan}\label{test-par-la-methode-de-breush-pagan}

\begin{verbatim}
## Test de Breusch-Pagan :
\end{verbatim}

\begin{verbatim}
## 
##  studentized Breusch-Pagan test
## 
## data:  modele_mco
## BP = 16.733, df = 4, p-value = 0.002178
\end{verbatim}

Conclusion \n On peut conclure de ce test que l'on constate la présence
d'hétéroscedasticité

\subsubsection{}\label{section}

\end{document}
